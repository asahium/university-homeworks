\documentclass[11pt]{article}

\makeatletter
\newcommand{\skipitems}[1]{%
	\addtocounter{\@enumctr}{#1}%
}
\makeatother

\newcommand{\numpy}{{\tt numpy}}
\usepackage{amsfonts}
\usepackage{amsmath}
\usepackage{graphics}
\usepackage{tikz} 
\usepackage{amsthm,amstext,amsfonts,bm,amssymb}
\usepackage{graphicx}
\graphicspath{ {./images/} }
\usepackage{indentfirst}
\setlength{\parindent}{5ex}
\setlength{\parskip}{1em}
\usepackage[utf8x]{inputenc} 
\usepackage[russian]{babel}
\topmargin -.5in
\textheight 9in
\oddsidemargin -.25in
\evensidemargin -.25in
\textwidth 7in


\begin{document}
	
	\author{доброжелатель}
	\title{Хаотические системы и хаос}
	\date{}
	\maketitle
	
	\medskip
	
	Базовой концепцией европейской науки $XVIII-XIX$ века была позиция об абсолютной познавательности мира. В ее предельном выражении эта позиция дается максимой Лапласа. Считалось, что неспособность понять и предсказать процессы и явления является следствием ограниченности наших знаний и/или вычислительных ресурсов; что развитие науки в конечном  даст возможность предсказывать поведение систем на много ($\infty$) шагов вперед.
	
	В $XX$ веке выяснилось, что существуют системы, где число шагов вперед, на каторые мы можем получить прогноз, является принципиально ограниченным. Верхняя граница -- горизон прогнозирования. Для регулярных систем горизонт прогнозирования бесконечен, а для хаотических не выполняется максима Лапласа, и горизонт ограничен. Значение легко определить по наблюдениям.
	
	Горизонт прогнозирования -- не следствие незнания, но следствие принципиальных свойств системы.
	
	Хаотические системы -- системы сложные. При работе со сложными системами (социальными. экономическими, финансовыми ...) неизбежно столкновение с хаосом как проявлением сложности.
	
	Примеры хаотических систем: погода, финансовые рынки, биение человеческого сердца, ЭЦГ.
	
	Единого определения хаоса не существует. "Отпечатки хаоса", проявляющиеся во всех хаотических системах: 
	\begin{enumerate}
		\item Конечный горизонт прогнозирования. Рассмотрим систему обыкновенных диффуров:
		$$\begin{cases}
			\frac{\partial x_1}{\partial t} = f_1 (x_1, \dots, x_n)\\
			\dots\\
			\frac{\partial x_n}{\partial t} = f_1 (x_1, \dots, x_n)\\
		\end{cases}$$
		В векторной форме: $\frac{\partial X}{\partial t} = f(X )$
	\end{enumerate}
	Решение $x_0(t)$ устойчиво по Ляпунову, если $$ \forall \partial>0 \:\exists\: \epsilon(\partial)>0:|x_0(0)-\overline{x_0}(0)|>\partial \Rightarrow |x_0(t)-\overline{x_0}(t)|>\epsilon $$
	Рассматривать неустойчивые системы практически бессмысленно.
	
	Траектория неустойчива по Ляпунову, если сколь бы малым не было начального возмущения, существует момент времени $T:\:|x_0(T)-\overline{x_0}(T)|>\epsilon$, сколь бы большим $\epsilon$ не было.
	
	Если система неустойчива по Ляпунову, то исходная и возмущенная траектории расходятся экспоненциально, причем показатель экспоненты $\lambda$ -- старший показатель Ляпунова.
	$$ |x_0(t)-\overline{x_0}(t)|\sim e^{\lambda t},\:\lambda>0 $$
	
	$$ (*)\:\:\: U(t)=\partial e^{\lambda t}, \text{где } \partial \text{ -- ошибка в начальный момент времени}$$
	
	Все хаотические траектории являются неустойчивыми по Ляпунову. Для неустойчивых по Ляпунову траекторий верно $(*)$.
	$$ \lambda>0 \text{ для хаотических систем } T<+\infty $$
	$$ \lambda<0 \text{ для регулярных систем } T=+\infty $$
	$$ U(t)\le \epsilon_{max} $$
	$$ T=\frac{1}{\lambda}\ln \frac{\epsilon_{max}}{\partial} \text{ -- горизонт (предел) прогнозирования}$$
	
	Базовым элементом является значение старшего показателя Ляпунова ($\lambda$). (Prediction horizon $\rightarrow$ число шагов. The horizon of predictability $\rightarrow$ T горизонт прогнозирования)
	$$\lambda = 0 \text{ "квазипериодическое движение".}$$
	
	$ \dot x = f(x) \text{ -- исследуемая система, } x_0(t) \text{ -- некая траектори} $. Пусть точка $\dot U = A(t)U$ есть лианеризация системы в окрестности траектории $x_0(t)$.
	
	Старшим показателем Ляпунова является $\lambda_1 = \overline{\lim_{t\to + \infty}}\frac{1}{t} \ln || U(t)|| $. В большинстве случаев "верхний" предел можно убрать, оставив просто предел. Если исходная система неустойчива по Ляпунову, то $U\sim e^{\lambda t}$.
	
	В формальном определении показателя Ляпунова есть некое несоответствие -- старший показатель Ляпунова есть базовая характеристика, но в определении фигурирует конкретная траектория $x_0(t)$. Это мнимое противоречие разрешается сложной теоремой: Мультипликативная эриодическая теорема. Ее базовый смысл -- показатель Ляпунова может принимать ровно $n$ значений $(\lambda_1, \dots, \lambda_n)$, причем случай общего положения является $\lambda_1 = \max \{ (\lambda_1, \dots, \lambda_n)\}$, который называется старший показатель Ляпунова. $\{\lambda_1, \dots, \lambda_n\}$ -- Ляпуновский спектр. Других значений показатель Ляпунова принимать не может. Мера тех точек, для которых не выполняется общее положение, равна 0. Мы должны принять меры, чтобы получить другие значения Ляпуновского спектра. Для регулярных систем $0>\lambda_1>\lambda_2>\dots$. Для хаотических систем типичной ситуацией является $\lambda_1>0, \lambda_2 = 0, \lambda_3>\lambda_4>\dots$. Если строго положительными являются несколько показателей, то говорят о гиперхаосе (гиперхаотических системах).
	
	Все алгоритмы оценки старшего показателя Ляпунова по наблюдаемому временному ряду базируются на том, что если наблюдаемая траектория (временной ряд) -- представляет собой движение вокруг странного аттрактора, то система с неизбежностью посещает одни и те же области фазового пространства. Если наблюдаемый временной ряд достаточно большой, в нем будут похожие участки (теорема Пуанкаре о возвращениях).
	
	Если мы наблюдаем во временном ряде два похожих участка, разнесенных на длительное время, мы оказываемся в условиях наблюдения неустойчивости по Ляпунову.
	
	
	
	
	
	
	
	
	
\end{document}