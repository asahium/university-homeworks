\documentclass[11pt]{article}

\makeatletter
\newcommand{\skipitems}[1]{%
	\addtocounter{\@enumctr}{#1}%
}
\makeatother

\newcommand{\numpy}{{\tt numpy}}
\usepackage{amsfonts}
\usepackage{amsmath}
\usepackage{graphics}
\usepackage{tikz} 
\usepackage{amsthm,amstext,amsfonts,bm,amssymb}
\usepackage{graphicx}
\graphicspath{ {./images/} }
\usepackage{indentfirst}
\setlength{\parindent}{5ex}
\setlength{\parskip}{1em}
\usepackage[utf8x]{inputenc} 
\usepackage[russian]{babel}
\topmargin -.5in
\textheight 9in
\oddsidemargin -.25in
\evensidemargin -.25in
\textwidth 7in


\begin{document}
	
	\author{Биктимиров Данила, группа 204}
	\title{ДЗ 6}
	\date{}
	\maketitle
	
	\medskip
	
	\begin{enumerate}
		
		\item Пусть $x<y$. Тогда $x\le \frac13$ и $\frac23\le y$, чтобы принимались сигналы от концов, а также расстояние между точками не больше $\frac23$, чтобы отрезок между ними покрывался. Иными словами, $y−x\le\frac23$. Отсюда имеем для этого случая пределы интегрирования $0\le x\le\frac13$, $\frac23\le y\le x+\frac23$. Берём удвоенный интеграл от совместной плотности (произведения плоскостей), так как есть ещё симметричный случай $y<x$.
		
		Получается
		$$8\int_{0}^{\frac{1}{3}}xdx\int_{\frac{2}{3}}^{x+\frac{2}{3}}ydy=\frac{19}{243}$$
		
		\item Для точки X на сфере мы будем использовать X'для обозначения точки, противоположной X. Это точка, которая находится дальше всего от X (диаметральная ей).
		
		После того, как выбраны три точки A, B, C, область, в которой нужно выбрать D, чтобы ABCD (покрытие) содержал O, представляет собой сферический треугольник A’B’C’, противоположный ABC. Следовательно, вероятность успеха - это просто ожидаемая площадь (сферического) треугольника A'B'C', нормализованная так, чтобы поверхность сферы имела площадь 1. Это явно совпадает с ожидаемой площадью ABC, и фактически это также ожидаемая область A'BC, A'BC'и так далее, поскольку все эти треугольники охватываются тремя равномерно выбранными точками на сфере. Сейчас существует 8 таких треугольников, и их общая площадь равна 1, поэтому ожидаемая площадь каждого из них составляет $\frac18$.
		
		Ответ, таким образом, $\frac18$.
		
	\end{enumerate}
\end{document}