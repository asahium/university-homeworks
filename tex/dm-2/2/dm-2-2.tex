\documentclass[11pt]{article}

\makeatletter
\newcommand{\skipitems}[1]{%
	\addtocounter{\@enumctr}{#1}%
}
\makeatother

\newcommand{\numpy}{{\tt numpy}}
\usepackage{amsfonts}
\usepackage{amsmath}
\usepackage{graphics}
\usepackage{amsthm,amstext,amsfonts,bm,amssymb}
\usepackage{graphicx}
\graphicspath{ {./images/} }
\usepackage{indentfirst}
\setlength{\parindent}{5ex}
\setlength{\parskip}{1em}
\usepackage[utf8x]{inputenc} 
\usepackage[russian]{babel}
\topmargin -.5in
\textheight 9in
\oddsidemargin -.25in
\evensidemargin -.25in
\textwidth 7in


\begin{document}
	
	\author{Биктимиров Данила, группа 204}
	\title{ДЗ 2}
	\date{}
	\maketitle
	
	\medskip
	
	\begin{enumerate}
		
		\item Покажем перечислимое, но не разрешимое множества. Берём любой известный нам пример (скажем, множество номеров машин Тьюринга, останавливающихся за конечное время). Тогда за $f(n)$ принимаем номер машины, которое перечисляющее устройство печатает на $n$-м шаге.
		\item Рассмотрим какую-нибудь невычислимую функцию g одной переменной, стремящуюся к бесконечности (получить её легко, взяв любую невычислимую и прибавив $x$). Построим по неё функцию двух переменных: $f(x,y)=1$ при $y<=g(x)$ и $f(x,y)=0$ при $y > g(x)$.
		
		Если бы $f$ была вычислима, мы могли бы по каждому x найти максимальное y, для которого $f(x,y)=1$. Понятно, что такое y равно $g(x)$, то есть получился бы алгоритм вычисления функции $g$.
		
		С другой стороны, при фиксированном $x$ в качестве сечения получается функция от $y$, которая почти всюду (всюду, кроме конечного числа номеров) равна нулю, а такая функция вычислима. Аналогично, если мы зафиксируем $y$, то в сечении получится функция от $x$, почти всюду равная 1. Она также вычислима.
		
		\item Допустим, что $f(x)$ вычислима при помощи программы с номером $p$. Тогда $f(x)=U(p,x)$ для всех $x$. В частности, $f(p)=U(p,p)$. Однако это противоречит определению функции $f$ как в случае $U(p,p)=2020$, так и в противном случае.
		\item
		\item Приведу в пример функцию Радо, так же известную как $BB(n)$ — функция от натурального аргумента $n$, равная максимальному числу шагов, которое может совершить программа длиной $n$ символов и затем остановиться. Докажем, что для любой вычислимой функции $f(n)$ функция $BB(n)$ будет превышать ее значение (за исключением конечного множества значений числа $n$.
		
		Пусть $f(n)$ представлена своим кодом. Для каждого $n$ определим программы вида:
		
		$p_n():$
		
		$k =$ десятичная запись числа $n$
		
		$m = f(k)$
		
		$for \: i = 1\: to \: m + 1 $
		
		шаг программы
		
		Каждая такая программа делает как минимум $f(n) + 1$ шагов.
		Так как мы рассматриваем $n$ в десятичной записи, то длина $p_n$ будет равна $\lg n + const$, где $const$ {{---}} длина кода без десятичной записи $n$. Пусть $n_0$ {{---}} решение уравнения $\lg n + const = n$. Тогда для всех натуральных $ n > \left \lceil n_0 \right \rceil $ будет выполнено неравенство: $ n > len(p_n) \Rightarrow BB(n) \geqslant BB(len(p_n)) > m = f(n) $. Данный переход корректен, так как мы доказали, что $BB(n)$ {{---}} монотонно возрастающая функция. Так как $n_0$ конечно, то мы всегда можем найти такие значения $n$, при которых будет выполняться полученное неравенство. Отсюда следует, что утверждение доказано.
		\item
		\item Если $A$ является $m$-сводимым к $N$, то $x \in A \Leftrightarrow f(x) \in N$. Последнее верно всегда. Значит, $A=N$. Во втором случае когда $N \le_m B$, условие $f(x) \in B$ всегда верно. Из определения $m$-сводимости мы получаем, что $f(x)$ всегда должно лежать в $B$. Пусть множество $B$ - непустое, и $f(x) = const =$ любой его элемент. Это значит, что B является непустым
		\item Вместо подмножеств $N$ будем рассматривать подмножества счётного множества $N \times Z$, которое равномощно $N$. Построим там подходящую цепочку, а потом перенесём её в $N$ посредством вычислимой биекции.
		
		В качестве $A(2k)$ берём $N \times \{i \in Z | i\le k\}$. Очевидно, все такие множества перечислимы. При переходе от $A(2k)$ к $A(2k+2)$ мы добавляем одну копию $N$, а именно, $N \times {2k+2}$. Зафиксируем какое-то одно неперечислимое подмножество $X < N$, и положим $A(2k+1) = A(2k) \cup (X \times {2k+2})$. Ясно, что оно будет неперечислимо.
		
		\item
		\item Рассмотрим перечислимое, но не разрешимое множество $X$. Для него существует тотальная вычислимая функция $f:N\to N$ такая, что $f(N)=X$. В качестве B возьмём множество всех нечётных чисел. Оно разрешимо. В качестве $A$ возьмём множество всех чисел вида $2^{f(n)}(2n-1)$. Оно разрешимо, так как всякое натуральное число m однозначно представимо в виде произведения степени двойки и нечётного числа. Представляя $m=2^s(2n-1)$ в таком виде, видим, что $s=f(n)$.
		
		В то же время, множество $A/B$ неразрешимо, так как степень двойки $2^k$ принадлежит этому множеству тогда и только тогда, когда k имеет вид $f(n)$, однако $X=f(N)$ у нас неразрешимо.
		
	\end{enumerate}
\end{document}