\documentclass[11pt]{article}

\makeatletter
\newcommand{\skipitems}[1]{%
	\addtocounter{\@enumctr}{#1}%
}
\makeatother

\newcommand{\numpy}{{\tt numpy}}
\usepackage{amsfonts}
\usepackage{amsmath}
\usepackage{graphics}
\usepackage{amsthm,amstext,amsfonts,bm,amssymb}
\usepackage{graphicx}
\graphicspath{ {./images/} }
\usepackage{indentfirst}
\setlength{\parindent}{5ex}
\setlength{\parskip}{1em}
\usepackage[utf8x]{inputenc} 
\usepackage[russian]{babel}
\topmargin -.5in
\textheight 9in
\oddsidemargin -.25in
\evensidemargin -.25in
\textwidth 7in


\begin{document}
	
	\author{Биктимиров Данила, группа 204}
	\title{ДЗ 7}
	\date{}
	\maketitle
	
	\medskip
	
	\begin{enumerate}
		
		\item да т.к $\arctg\frac{x}{n^2}arctg\sim \frac{x}{n^2}$ то ряд сходится, кроме того ряд производных $\sum_{n=1}^{\infty} \frac{n^2}{n^4+x^2}$ по признаку Вейерштрасса сходитя равномерно. Мажоранта здесь $\frac{1}{n^2}$, то по признаку Лейбница почленное дифференцирование допустимо.
	\end{enumerate}
\end{document}