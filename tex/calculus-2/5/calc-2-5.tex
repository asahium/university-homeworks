\documentclass[11pt]{article}

\makeatletter
\newcommand{\skipitems}[1]{%
	\addtocounter{\@enumctr}{#1}%
}
\makeatother

\newcommand{\numpy}{{\tt numpy}}
\usepackage{amsfonts}
\usepackage{amsmath}
\usepackage{graphics}
\usepackage{amsthm,amstext,amsfonts,bm,amssymb}
\usepackage{graphicx}
\graphicspath{ {./images/} }
\usepackage{indentfirst}
\setlength{\parindent}{5ex}
\setlength{\parskip}{1em}
\usepackage[utf8x]{inputenc} 
\usepackage[russian]{babel}
\topmargin -.5in
\textheight 9in
\oddsidemargin -.25in
\evensidemargin -.25in
\textwidth 7in


\begin{document}
	
	\author{Биктимиров Данила, группа 204}
	\title{ДЗ 4}
	\date{}
	\maketitle
	
	\medskip
	
	\begin{enumerate}
		
		\item \begin{enumerate}
			\item $$\int \sin^n(x)dx = \left| u(x) = \sin^{n-1}x \:\: v(x)= -\cos(x) \right| = -\frac{\sin^{n-1}(x)\cos(x)}{n} + \int \frac{n-1}{n}\sin^{n-2}(x)dx = $$
			$$= -\frac{\sin^{n-1}(x)\cos(x)}{n} + \frac{n-1}{n} \int \sin^{n-2}(x)dx$$
			$$\int_{0}^{\frac{\pi}{2}}\sin^n (x)dx = \frac{-\sin^{n-1}(x)\cos (x)}{n}\Bigr|_0^{\frac{\pi}{2}} +  \frac{n-1}{n} \int_0^{\frac{\pi}{2}} \sin^{n-2}(x)dx = $$
			$$=\frac{n-1}{n} \int_0^{\frac{\pi}{2}} \sin^{n-2}(x)dx, \text{ при } n\ge 2$$
			Дальше по индукции
			$$n = 1\:\:\: \int_{0}^{\frac{\pi}{2}}\sin (x)dx = \cos(\frac{\pi}{2})-\cos 0 = 1 = \frac{0!!}{1!!}$$
			$$n = 2\:\:\: \int_{0}^{\frac{\pi}{2}}\sin^2 (x)dx = \frac{\pi}{4}- 0 -\frac{\sin \pi}{4} + \frac{\sin 0}{4}= \frac{\pi}{4} \cdot \frac{0!!}{1!!}$$
			Пусть верно для $k = 2n-1$
			$$\int_{0}^{\frac{\pi}{2}}\sin^{2n+1}(x)dx = \frac{2n}{2n+1} \int_{0}^{\frac{\pi}{2}}\sin^{2n-1}dx = \frac{2n}{2n+1}\cdot \frac{(2n-2)!!}{(2n-1)!!} = \frac{2n!!}{(2n+1)!!}$$
			Пусть верно для $k = 2n-2$
			$$\int_{0}^{\frac{\pi}{2}}\sin^{2n}(x)dx = \frac{2n-1}{2n} \int_{0}^{\frac{\pi}{2}}\sin^{2n-2}dx = \frac{2n-1}{2n}\cdot \frac{\pi}{2} \cdot \frac{(2n-3)!!}{(2n-2)!!} = \frac{\pi}{2} \cdot \frac{(2n-1)!!}{2n!!}$$
			\item При $0\le x \le \frac{\pi}{2}$ верно $0\le \sin x \le1$ и значит $\sin^{2n+2}(x)\le \sin^{2n+1}(x)\le \sin^{2n}(x)$.
			
			Отсюда $$ \int_{0}^{\frac{\pi}{2}}\sin^{2n+2}(x)dx= \frac{2n+1}{2n+2}\int_{0}^{\frac{\pi}{2}}\sin^{2n}(x)dx \le \int_{0}^{\frac{\pi}{2}} \sin^{2n+1}(x)dx \le \int_{0}^{\frac{\pi}{2}}\sin^{2n}(x)dx$$
			Получим $$ \frac{2n+1}{2n+2} \le \frac{\int_{0}^{\frac{\pi}{2}} \sin^{2n+1}(x)dx}{\int_{0}^{\frac{\pi}{2}}\sin^{2n}(x)dx} \le 1$$ Ну и по теореме о 2 милиционерах получим желаемое.
			\item $$ \frac{\int_{0}^{\frac{\pi}{2}} \sin^{2n+1}(x)dx}{\int_{0}^{\frac{\pi}{2}}\sin^{2n}(x)dx}= \frac{\frac{(2n)!!}{(2n+1)!!}}{\frac{\pi}{2}\cdot \frac{(2n-1)!!}{(2n)!!}} \to 1$$
			Тогда $$ \lim_{n\to \infty} \left(\frac{(2n)!!}{(2n-1)!!} \cdot \frac{(2n)!!}{(2n+1)!!} \right) = \frac{\pi}{2} $$
			$$ (2n)!! = 2^n\cdot n!,\:\:\:(2n+1)!!  = \frac{(2n+1)!!}{(2n)!!} = \frac{(2n+1)!}{2^n\cdot n!} $$
			Имеем $$ \lim_{n\to \infty} \frac{(2n)!!(2n!!)}{(2n-1)!!(2n+1)!!} = \lim_{n\to \infty} \frac{2^{4n-1}\cdot (n!)^3\cdot (n-1)!}{(2n-1)!(2n+1)!} = $$
			$$ =\lim_{n\to \infty} \frac{2^{4n-1}\cdot C^4n\sqrt{n^2-n}\left(\frac{n}{e}\right)^{3n}\left( \frac{n-1}{e}\right)^{n-1} }{C^2\sqrt{4n^2-1}\left(\frac{2n-1}{e}\right)^{2n-1}\cdot \left(\frac{2n+1}{e}\right)^{2n+1}}=$$
			$$ = \lim_{n\to \infty} \frac{2^{4n-1}\cdot C^4}{C^2\cdot 2\cdot 2^{2n-1}\cdot 2^{2n+1}} = \lim_{n\to \infty} \frac{C^2}{4} = \frac{C^2}{4}$$
			И $$ \frac{C^2}{4} = \frac{\pi}{2} \Rightarrow C=\sqrt{2\pi} $$ Победа!
		\end{enumerate}
	
	\item \begin{enumerate}
		\item 
		
		$$\prod_{n=1}^{\infty} \left(1 + \frac{(-1)^{\frac{n(n-1)}{2}}}{n} \right) =
		\sum_{n=1}^{\infty} \operatorname{ln}\left(1 + \frac{(-1)^{\frac{n(n-1)}{2}}}{n} \right) = $$
		$$ = \sum_{n=1}^{\infty} \left( \frac{(-1)^{\frac{n(n-1)}{2}}}{n}\right) + o \left(\frac{1}{n^2} \right) \text{ - абсолютно очевидно расходится}$$
		
		$$\sum_{n=1}^{\infty} \left( \frac{(-1)^{\frac{n(n-1)}{2}}}{n}\right) = \sum_{n=1}^{\infty} \left(\frac{1}{4n} -\frac{1}{4n+1} -\frac{1}{4n+2} +\frac{1}{4n+3}  \right) =$$ $$ =\sum_{n=1}^{\infty} \frac{1}{4n(4n+1)} - \frac{1}{(4n+2)(4n+3)} = \sum_{n=1}^{\infty} \frac{16n^2 + 20n+6 -16n^2 -4n}{4n(4n+1)(4n+2)(4n+3)} = \sum_{n=1}^{\infty} \frac{16n +6}{o(n^4)} \text{ - сходится}$$
		
		\item    
		
		$$\prod_{n=1}^{\infty} \left(
		\frac{\sqrt{n}}{\sqrt{n} + (-1)^{n-1}} \right) = \prod_{n=1}^{\infty} \left( 1 - 
		\frac{(-1)^{n-1}}{\sqrt{n} + (-1)^{n-1}} \right) = \sum_{n=1}^{\infty} \operatorname{ln} \left( 1 + 
		\frac{(-1)^{n}}{\sqrt{n} + (-1)^{n-1}} \right) = \sum_{n=1}^{\infty} \frac{(-1)^n}{\sqrt{n}} - \frac{1}{n} = $$ $$= \underbrace{\sum_{n=1}^{\infty} \frac{(-1)^n}{\sqrt{n}}}_{\text{сходится по Лейбницу}} -\underbrace{\sum_{n=1}^{\infty} \frac{1}{n}}_{\text{расходится}} \Rightarrow \text{ ряд сходится условно и абсолютно}$$
	\end{enumerate}
	\item  $$\sum_{n=1}^{\infty}\frac{(-1)^{n-1}}{n^{\alpha}} \cdot \sum_{n=1}^{\infty}\frac{(-1)^{n-1}}{n^{\beta}} =\sum_{n=1}^{\infty} c_n $$

	$$c_n = \sum_{k=1}^{n} a_k \cdot b_{n+1-k} = \sum_{k=1}^{n} \frac{(-1)^{k-1}}{k^{\alpha}} \cdot \frac{(-1)^{n+1 - k-1}}{(n+1-k)^{\beta}} = (-1)^{n+1} \sum_{k=1}^{n} \frac{1}{k^{\alpha}(n+1-k)^{\beta}}$$

	$$\forall k \in [1;n] : 0 < k^{\alpha}(n+1-k)^{\beta} \leq n^{\alpha} n^{\beta} =  n^{\alpha + \beta}$$

	$$\sum_{k=1}^{n} \frac{1}{k^{\alpha}(n+1-k)^{\beta}} \geq \frac{n}{n^{\alpha + \beta}} \cdot n^{1- \alpha - \beta}$$

Очевидно, общий член не стремится к 0, значит, ряд расходится.
		
	\end{enumerate}
\end{document}